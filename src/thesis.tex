%--------------------------------------------------------------------%
%
% Berkas utama templat LaTeX.
%
% author Petra Barus, Peb Ruswono Aryan
% updated by Dionesius Agung (2020)
% adapted by Irfan Tito Kurniawan (2020)
% adapted for ET by Rama Rahardi (2021)
%--------------------------------------------------------------------%
%
% Berkas ini berisi struktur utama dokumen LaTeX yang akan dibuat.
%
%--------------------------------------------------------------------%
\documentclass[12pt, a4paper, onecolumn, oneside, final]{report}

%-------------------------------------------------------------------%
%
% Konfigurasi dokumen LaTeX untuk laporan tesis IF ITB
%
% @author Petra Novandi
% updated by Dionesius Agung (2020)
% adapted by Irfan Tito Kurniawan (2020)
% adapted for ET by Rama Rahardi (2021)
%-------------------------------------------------------------------%
%
% Berkas asli berasal dari Steven Lolong
%
%-------------------------------------------------------------------%

%%%%%%%%%%%%%%%%%%%%%%%%%%%%%%%%%%%%
%  DOCUMENT LAYOUT AND FORMATTING  %
%%%%%%%%%%%%%%%%%%%%%%%%%%%%%%%%%%%%
% Document layout
\usepackage[top=3cm,bottom=3cm,left=4cm,right=3cm,a4paper]{geometry}

% Uncomment these two packages if you want to use
%  the Times New Roman font for your document
\usepackage{mathptmx}
\usepackage{newtxtext}

% Judul bahasa Indonesia
\usepackage[bahasa]{babel}

% Spacing 1.5
\usepackage{setspace}
\renewcommand{\baselinestretch}{1.5}

% Prevent overfull (or underfull) if possible
%\setlength{\emergencystretch}{25pt}

% Avoid widow and orphan lines if possible
\widowpenalty500
\clubpenalty10000

% Hyphenation penalty
\include{hyphenation-id}
\hyphenpenalty=10000
\tolerance=1
\sloppy

%%%%%%%%%%%%%%%%%%%%%%%%%%%%%%%
%  BIBLIOGRAPHY AND CITATION  %
%%%%%%%%%%%%%%%%%%%%%%%%%%%%%%%

%%%%%%%%%%%%%%
%  PACKAGES  %
%%%%%%%%%%%%%%
\usepackage[utf8]{inputenc}
\usepackage{graphicx}
\usepackage{titling}
\usepackage{blindtext}
\usepackage{sectsty}
\usepackage{chngcntr}
\usepackage{etoolbox}
\usepackage{hyperref}
\usepackage{titlesec}
\usepackage{parskip}
\usepackage{booktabs}
\usepackage{tabularx}
\usepackage[chapter]{algorithm}
\usepackage{algpseudocode}
\usepackage{comment}
\usepackage{cite}

%%%%%%%%%%%%%%$$%%%%%%%%%
%  CHAPTER AND SECTION  %
%%%%%%%%%%%%%%%%$$%%%%%%%
% Format judul bab
\chapterfont{\centering \large}
\titleformat{\chapter}[hang] %[display]
{\large\centering\bfseries}
{\chaptertitlename\ \Roman{chapter} }{0em}
{\large\bfseries\MakeUppercase}
\titlespacing*{\chapter}
{0pt}
{-1.5\baselineskip}
{1\baselineskip}

% Format judul section (dan sub(sub)section)
\titleformat*{\section}{\bfseries\normalsize}
\titleformat*{\subsection}{\bfseries\normalsize}
\titleformat*{\subsubsection}{\bfseries\normalsize}
\titlespacing*{\section}{0pt}{2ex}{0pt}
\titlespacing*{\subsection}{0pt}{2ex}{0pt}
\titlespacing*{\subsubsection}{0pt}{2ex}{0pt}

% Kedalaman hierarki section (paling dalam subsubsection)
\setcounter{secnumdepth}{3}


%%%%%%%%%%%%%%%%%%%%%%%%%%%%%%%%%%%%%%%%%%%%%%%%%%
%  TABLE OF CONTENTS, LISTS OF FIGURES & TABLES  %
%%%%%%%%%%%%%%%%%%%%%%%%%%%%%%%%%%%%%%%%%%%%%%%%%%
\usepackage[titles]{tocloft}
\usepackage[titletoc]{appendix}
\usepackage{tocbibind}

% Kedalaman hierarki maksimum ToC
% (yang masuk ToC hanya sampai subsection: I.1.1.)
\setcounter{tocdepth}{2}

% Hilangkan gap antar-bab di ToC
\setlength{\cftbeforechapskip}{0pt}

% Tambah kata "BAB" sebelum nomor bab di daftar isi
% TODO: still problematic when used with list of appendices (uncomment these 4 following lines to reproduce the problem)
% \renewcommand{\cftchappresnum}{BAB~} % BAB before number in ToC
% \newlength{\mylen} % a scratch length
% \settowidth{\mylen}{\bfseries\cftchappresnum\cftchapaftersnum} % extra space
% \addtolength{\cftchapnumwidth}{\mylen} % add the extra space

% Pisah daftar lampiran dari ToC
%%%
\renewcommand{\appendixtocname}{Daftar Lampiran}

\makeatletter
\let\oldappendix\appendices

\renewcommand{\appendices}{%
	\clearpage
	% From now, everything goes to the app file and not to the toc
	\let\tf@toc\tf@app
	\addtocontents{app}{\protect\setcounter{tocdepth}{1}}
	\immediate\write\@auxout{%
		\string\let\string\tf@toc\string\tf@app^^J
	}
	\oldappendix
}%

\newcommand{\listofappendices}{%
	\begingroup
	\renewcommand{\contentsname}{\appendixtocname}
	\let\@oldstarttoc\@starttoc
	\def\@starttoc##1{\@oldstarttoc{app}}
	% Reusing the code for \tableofcontents with different
	%   \contentsname and different file handle app
	\tableofcontents
	\endgroup
}
\makeatother
%%%

% Hilangkan gap antara entri gambar & tabel antarbab di daftar tabel 
% dan daftar gambar (hanya terlihat kalau ada gambar/tabel di >1 bab)
\newcommand*{\noaddvspace}{\renewcommand*{\addvspace}[1]{}}
\addtocontents{lof}{\protect\noaddvspace}
\addtocontents{lot}{\protect\noaddvspace}

%%%%%%%%%%%%%%%%%%%%%%%%%%%%%%%%%%%%%%%%%
%  FLOATS: FIGURES, TABLES, ALGORITHMS  %
%%%%%%%%%%%%%%%%%%%%%%%%%%%%%%%%%%%%%%%%%
% Before:
% ---
% Counter untuk figure dan table.
% \counterwithin{figure}{section}
% \counterwithin{table}{section}
% ---

\usepackage[labelsep=period,
justification=justified,
format=hang]{caption}
\usepackage[labelformat=simple]{subcaption}
%% Hack subfigure cross-ref agar pakai tanda kurung
%%   e.g. Gambar II.2(a), bukan Gambar II.2a
%% (method recommended in subcaption package documentation)
\renewcommand\thesubfigure{(\alph{subfigure})}

% Counter untuk gambar dan tabel
\renewcommand*{\thefigure}{\Roman{chapter}-\arabic{figure}}
\renewcommand*{\thetable}{\Roman{chapter}-\arabic{table}}

% Jarak spasi antara float dengan teks utama
\captionsetup[figure]{belowskip=-1em}
\captionsetup[subfigure]{belowskip=0pt}
\setlength{\textfloatsep}{2\baselineskip}
\setlength{\intextsep}{2\baselineskip}

% Spasi single di environment table
\AtBeginEnvironment{table}
{\renewcommand{\baselinestretch}{1.0}}

% Font lebih kecil untuk tabel
\AtBeginEnvironment{tabular}
{\small}

% Spasi single di environment algorithm
\AtBeginEnvironment{algorithm}
{\renewcommand{\baselinestretch}{1.0}}

% Rename "Algorithm" into "Algoritma"
\makeatletter
\renewcommand*{\ALG@name}{Algoritma}
\newcommand{\algorithmname}{\ALG@name}
\makeatother


%%%%%%%%%%%%%%%%%%%%%%%%%
%  MATHS AND EQUATIONS  %
%%%%%%%%%%%%%%%%%%%%%%%%%
\usepackage{amsmath}
\usepackage{amsfonts}
\usepackage{mathtools}

% Counter untuk equation
\renewcommand*{\theequation}{\Roman{chapter}.\arabic{equation}}

% Allow page breaks on long equations
\allowdisplaybreaks[1-4]

% Operator dan notasi custom tambahan
% contoh: argmin dan argmax
\DeclareMathOperator*{\argmax}{argmax}
\DeclareMathOperator*{\argmin}{argmin}
% contoh: notasi bayes p(x | y)
\newcommand{\bayes}[2]{p(#1 \mid #2)\xspace}

%%%%%%%%%%%%%%%%%
%  GANTT CHART  %
%%%%%%%%%%%%%%%%%
\usepackage{changepage}
\usepackage{pgfgantt}
\ganttset{calendar week text={\currentweek}}

%%%%%%%%%%%%%%
%  DIAGRAMS  %
%%%%%%%%%%%%%%
\usepackage{tikz}
\usetikzlibrary{shapes, arrows, fit}
\tikzstyle{block} = [rectangle, draw, minimum height=3em, minimum width=6em, text centered]
\tikzstyle{pinstyle} = [pin edge={to-, thin, black}]
\tikzstyle{input} = [coordinate]
\tikzstyle{output} = [coordinate]

%%%%%%%%%%%%%%%%%%%%%%%%%%%%%%%%
%  GLOSSARY AND ABBREVIATIONS  %
%%%%%%%%%%%%%%%%%%%%%%%%%%%%%%%%
% Load package acronym dan indexonlyfirst untuk hanya 
% menunjukkan kemunculan pertama singkatan
\usepackage[nomain, abbreviations, indexonlyfirst]{glossaries-extra}

% Buat glossary baru khusus untuk lambang
\newglossary[slg]{symbols}{syi}{syg}{Daftar Lambang}

% Buat glossary
\makeglossaries

% Hilangkan judul dari glossary
\renewcommand{\glossarysection}[2][]{}

% Style glossary untuk Daftar Singkatan
\newglossarystyle{daftarsingkatan}{
	% Dasarkan style pada style long3colheader
	\setglossarystyle{long3colheader}
	
	\renewenvironment{theglossary}{\begin{longtable}{p{2cm}p{\glsdescwidth}p{\glspagelistwidth}}}{\end{longtable}}
	
	% Ganti header glossary
	\renewcommand{\glossaryheader}{
		\centering \textbf{SINGKATAN} &
		\centering \textbf{NAMA} &
		\centering \textbf{KEMUNCULAN PERTAMA}
		\endhead
	}
	
	% Ganti lebar kolom glossary
	\renewcommand{\glsdescwidth}{7cm}
	\renewcommand{\glspagelistwidth}{4cm}
	
	% Buat line break dalam sel menjadi 1 spasi
	\renewcommand{\baselinestretch}{1.0} 
	\renewcommand{\arraystretch}{1.5}
	\selectfont
	
	% Ganti isi glossary menjadi singkatan - deskripsi - kemunculan pertama
	\renewcommand{\glossentry}[2]{
		\glsentryitem{##1}\glstarget{##1}{\glossentryname{##1}} 
		& \glossentrydesc{##1}
		& \centering ##2
		\tabularnewline
	}
	
	\renewcommand{\glsgroupskip}{}
}

% Style glossary untuk Daftar Lambang
\newglossarystyle{daftarlambang}{
	% Dasarkan style pada style long3colheader
	\setglossarystyle{long3colheader}
	% Hack untuk mengubah lebar kolom lambang
	\renewenvironment{theglossary}{\begin{longtable}{p{2cm}p{\glsdescwidth}p{\glspagelistwidth}}}{\end{longtable}}
	% Ganti header glossary
	\renewcommand{\glossaryheader}{
		\centering \textbf{LAMBANG} &
		\centering \textbf{NAMA} &
		\centering \textbf{KEMUNCULAN PERTAMA}
		\endhead
	}
	
	% Ganti lebar kolom glossary
	\renewcommand{\glsdescwidth}{7cm}
	\renewcommand{\glspagelistwidth}{4cm}
	
	% Buat line break dalam sel menjadi 1 spasi
	\renewcommand{\baselinestretch}{1.0} 
	\renewcommand{\arraystretch}{1.5}
	\selectfont
	
	% Ganti isi glossary menjadi lambang - deskripsi - kemunculan pertama
	\renewcommand{\glossentry}[2]{
		\glsentryitem{##1}\glstarget{##1}{\glossentryname{##1}} 
		& \glossentrydesc{##1}
		& \centering ##2
		\tabularnewline
	}
	
	\renewcommand{\glsgroupskip}{}
}

\makeatletter

\makeatother

\begin{document}

    %Basic configuration
    \title{Pengembangan Desain Susunan Antena untuk Estimasi \textit{Direction of Arrival} (DoA) dengan Kemampuan \textit{Beamforming} Sederhana}
    \date{\today}
    \author{
        Rama Rahardi \\
        NIM: 18117026
    }

    \pagenumbering{roman}
    \setcounter{page}{0}

    \clearpage
\pagestyle{empty}

% Setting margin for cover page
\newgeometry{top=3cm,bottom=3cm,left=3cm,right=3cm}

% Use Times font for cover page as per the thesis document guidelines
%{\fontfamily{ptm}\selectfont%
\begin{center}
    
    \smallskip
	\renewcommand{\baselinestretch}{1}
	
    \large{\bfseries \MakeUppercase{\thetitle}}
    \\[5\baselineskip]

    \large{\bfseries TUGAS AKHIR}
    \\[\baselineskip]
	
    \normalsize{ \bfseries
    	Karya tulis sebagai salah satu syarat\\
    	untuk memperoleh gelar Sarjana dari\\
    	Institut Teknologi Bandung
	}
    \\[3\baselineskip]

    \normalsize{ \bfseries Oleh\\}
    \large{ 
    	\bfseries \MakeUppercase{\theauthor}\\
    	(Program Studi Teknik Telekomunikasi)
	}

    \vfill
    \begin{figure}[h]
        \centering
      	\includegraphics[height=3.5cm,keepaspectratio]{resources/cover-ganesha.jpg}
    \end{figure}
    \vfill

    \large{ \bfseries
	    \uppercase{
	        Institut Teknologi Bandung\\
	    }
    	Mei 2024
	}

\end{center}
%}%

\restoregeometry
\clearpage

    \clearpage
\chapter*{Abstrak}
\addcontentsline{toc}{chapter}{ABSTRAK}

\begin{center}
	\linespread{1}
	\large{\bfseries{
			\MakeUppercase\thetitle
		}
	}\\[1\baselineskip]
	\normalsize{Oleh\\}
	\large{ 
		\bfseries \theauthor\\
		(Program Studi Sarjana Teknik Biomedis)
	}\\[2\baselineskip]
\end{center}
\begin{spacing}{1.0}
	%taruh abstrak bahasa indonesia di sini
	\blindtext
	
	\blindtext
	\\[1.67\baselineskip]
	Kata kunci: pertama, kedua, ketiga.
\end{spacing}

\clearpage

	\clearpage
\chapter*{Abstract}
\addcontentsline{toc}{chapter}{\textit{ABSTRACT}}

\begin{center}
	\linespread{1}
	\large{\bfseries{
			\MakeUppercase{\textit{Judul dalam Bahasa Inggris}}
		}
	}\\[1\baselineskip]
	\normalsize{By\\}
	\large{ 
		\bfseries \theauthor\\
		(Undergraduate Program in Biomedical Engineering)
	}\\[2\baselineskip]
\end{center}
\begin{spacing}{1.0}
	%taruh abstrak bahasa inggris di sini bila diperlukan
	\itshape{
		\blindtext
		
		\blindtext
		\\[1.67\baselineskip]
		Keywords: first, second, third.
	}
\end{spacing}

\clearpage

    \input{chapters/approval-two-advisors}
    \clearpage
%\pagestyle{empty}

\begin{center}    
	\renewcommand{\baselinestretch}{1}
    \large{\bfseries \MakeUppercase{\thetitle}}
    \\[2\baselineskip]

\large{\bfseries HALAMAN PENGESAHAN}
    \\[\baselineskip]

    \normalsize{Oleh\\
    	\textbf{\theauthor}\\
    	\textbf{(Program Studi Teknik Telekomunikasi)}
    	\\[\baselineskip]
    	Institut Teknologi Bandung}
    \\[3\baselineskip]
    
    
    \normalsize{Menyetujui\\
    	Tim Pembimbing
    	\\[\baselineskip]
    	Tanggal \thedate}
    \\[5\baselineskip]
    
    \normalsize{%
    \setlength{\tabcolsep}{12pt}
    \begin{tabular}{c@{\hskip 0.5in}c}
        Ketua & Anggota \\
        & \\
        & \\
        & \\
        \rule{5cm}{0.4pt} & \rule{5cm}{0.4pt} \\
        Nama dan Gelar Pembimbing I & Nama dan Gelar Pembimbing II \\
        NIP. 123456789 & NIP. 123456789 \\
    \end{tabular}
    }

\end{center}
\clearpage

    \clearpage
\chapter*{Pedoman Penggunaan Tugas Akhir}
\addcontentsline{toc}{chapter}{PEDOMAN PENGGUNAAN TUGAS AKHIR}

Tugas Akhir Sarjana yang tidak dipublikasikan terdaftar dan tersedia di Perpustakaan Institut Teknologi Bandung, dan terbuka untuk umum dengan ketentuan bahwa hak cipta ada pada penulis dengan mengikuti aturan HaKI yang berlaku di Institut Teknologi Bandung. Referensi kepustakaan diperkenankan dicatat, tetapi pengutipan atau peringkasan hanya dapat dilakukan seizin penulis dan harus disertai dengan kaidah ilmiah untuk menyebutkan sumbernya.

Sitasi hasil penelitian Tugas Akhir ini dapat ditulis dalam bahasa Indonesia sebagai berikut:

\hangindent=1.27cm Rahardi, Rama. (\the\year): \textit{\thetitle}, Tugas Akhir Program Sarjana, Institut Teknologi Bandung\\

dan dalam bahasa Inggris sebagai berikut:

\hangindent=1.27cm Rahardi, Rama. (\the\year): \textit{Biomedical Engineering Thesis Template}, Undergraduate Final Year Project, Institut Teknologi Bandung\\

Memperbanyak atau menerbitkan sebagian atau seluruh Tugas Akhir haruslah seizin Dekan Sekolah Teknik Elektro dan Informatika Institut Teknologi Bandung.

\clearpage
    \clearpage

\begin{center}
    \topskip0pt
    \vspace*{\fill}
    \textit{
    Jangan bingung\\
    Tidak usah repot-repot
    }
    \vspace*{\fill}
\end{center}

\clearpage

    \pagestyle{plain}

    % Frontmatter
    \chapter*{Kata Pengantar}
\addcontentsline{toc}{chapter}{KATA PENGANTAR}

%Gunakan bagian ini untuk memberikan ucapan terima kasih kepada semua pihak yang secara langsung atau tidak langsung membantu penyelesaian tugas akhir, termasuk pemberi beasiswa jika ada. Utamakan untuk memberikan ucapan terima kasih kepada tim pembimbing tugas akhir dan staf pengajar atau pihak program studi, bahkan sebelum mengucapkan terima kasih kepada keluarga. Ucapan terima kasih sebaiknya bukan hanya menyebutkan nama orang saja, tetapi juga memberikan penjelasan bagaimana bentuk bantuan/dukungan yang diberikan. Gunakan bahasa yang baik dan sopan serta memberikan kesan yang enak untuk dibaca. Sebagai contoh: “Tidak lupa saya ucapkan terima kasih kepada teman dekat saya, Tito, yang sejak satu tahun terakhir ini selalu memberikan semangat dan mengingatkan saya apabila lengah dalam mengerjakan Tugas Akhir ini. Tito juga banyak membantu mengoreksi format dan layout tulisan. Apresiasi saya sampaikan kepada pemberi beasiswa, Yayasan Beasiswa, yang telah memberikan bantuan dana kuliah dan biaya hidup selama dua tahun. Bantuan dana tersebut sangat membantu saya untuk dapat lebih fokus dalam menyelesaikan pendidikan saya. ....”. Ucapan permintaan maaf karena kekurangsempurnaan hasil Tugas Akhir tidak perlu ditulis.

Puji syukur kepada Tuhan Yang Maha Esa atas berkat dan karunia-Nya yang telah
memberikan kesempatan penulis untuk menyelesaikan salah satu kewajiban dalam
menempuh studi sarjana S1 pada Program Studi Teknik Telekomunikasi di Institut
Teknologi Bandung yaitu Tugas Akhir berjudul “\textbf{\thetitle}”.

Ucapan terima kasih dan rasa syukur juga tidak lupa disampaikan oleh penulis
kepada seluruh orang yang telah melancarkan dan membantu dalam pelaksanaan
Tugas Akhir yang telah diberikan baik dalam bentuk usaha, waktu, material dan
juga dukungan. Tanpa ada dukungan dari orang-orang tersebut, penulis tidak akan
mampu untuk menyelesaikan pengerjaan Tugas Akhir ini dengan baik. Maka
izinkanlah penulis menyampaikan rasa terima kasih kepada

\begin{enumerate}
    \item Orang Tua penulis yang selalu memberikan dukungan finansial maupun secara moral
    \item Bapak Prof. Ir. Hendrawan, M.Sc., Ph.D. selaku dosen pembimbing yang selalu membimbing dan membantu dalam pengerjaan tugas akhir ini
    \item Prof. Junchen Jiang dan Prof. Haryadi S. Gunawi selaku kolaborator yang selalu memberikan masukan
    \item Roy Huang selaku rekan kolaborator riset yang telah banyak membantu terkait hal teknis 
    \item Farhan Krishna selaku rekan TA yang kadang nyebelin
\end{enumerate}

Penulisan buku tugas akhir ini tidak akan bisa dilakukan tanpa adanya orang-orang
yang selalu membantu dalam penyelesaiannya. Penulis buku akhir in hanyalah
manusia yang tidak lepas dari kesalahan. Maka dari itu, penulis terbuka dan
menerima kritik, saran dan diskusi sebagai bahan perbaikan dan pembelajaran agar
penulis dapat menjadi pribadi yang lebih baik lagi kedepannya. Semoga Buku tugas
akhir yang penulis but mampu bermanfaat bagi pembaca, terutama teman-teman
pegiat telekomunikasi.

% \\[\baselineskip]
Bandung, \thedate \\[\baselineskip]
Penulis

    % Hacks to capitalize all chapter-level titles in ToC
    \renewcommand*\contentsname{DAFTAR ISI}
    \renewcommand*\appendixtocname{DAFTAR LAMPIRAN}
    \renewcommand*\listfigurename{DAFTAR GAMBAR DAN ILUSTRASI}
    \renewcommand*\listtablename{DAFTAR TABEL}
    \renewcommand*\bibname{DAFTAR PUSTAKA}

    % Lanjutan frontmatter
    \tableofcontents
    \listofappendices
    {%
		\let\oldnumberline\numberline%
		\renewcommand{\numberline}{\figurename~\oldnumberline}%
		\listoffigures%
	}
	{%
		\let\oldnumberline\numberline%
		\renewcommand{\numberline}{\tablename~\oldnumberline}%
		\listoftables%
	}
	\clearpage

\chapter*{Daftar Singkatan}
\addcontentsline{toc}{chapter}{DAFTAR SINGKATAN DAN LAMBANG}

% Singkatan yang digunakan
\newabbreviation{utc}{UTC}{Coordinated Universal Time}
\newabbreviation{adt}{ADT}{Atlantic Daylight Time}
\newabbreviation{est}{EST}{Eastern Standard Time}

\newglossaryentry{symb:pi}{name=$\pi$, description={Rasio keliling lingkaran terhadap diameternya}, type=symbols}
\newglossaryentry{symb:avogadro}{name=$N_A$, description={Bilangan Avogadro}, type=symbols}

\printglossary[style=daftarsingkatan]

\printglossary[type=symbols, style=daftarlambang]

\clearpage
    
    %----------------------------------------------------------------%
    % Konfigurasi Bab
    %----------------------------------------------------------------%
    %------------------------------------------------------%
    % Hack: 2 baris berikut dipindah ke chapter-1.tex
    %   previous method: nomor halaman sebelum BAB I jadi 0
    %------------------------------------------------------%
    % \pagenumbering{arabic}
    % \setcounter{page}{0}
    \renewcommand{\chaptername}{BAB}
    %\renewcommand{\thechapter}{\arabic{chapter}}
    %\renewcommand{\thechapter}{\Roman{chapter}}
    %----------------------------------------------------------------%

    %----------------------------------------------------------------%
    % Dafter Bab
    % Untuk menambahkan daftar bab, buat berkas bab misalnya `chapter-6` di direktori `chapters`, dan masukkan ke sini.
    %----------------------------------------------------------------%
    \chapter{PENDAHULUAN}
% Hack: gatau kenapa harus gini
\pagenumbering{arabic}
\setcounter{page}{1}

Bab Pendahuluan secara umum yang dijadikan landasan kerja dan arah kerja penulis tugas akhir, berfungsi mengantar pembaca untuk membaca laporan tugas akhir secara keseluruhan.

\section{Latar Belakang}

Latar Belakang berisi dasar pemikiran, kebutuhan atau alasan yang menjadi ide dari topik tugas akhir. Tujuan utamanya adalah untuk memberikan informasi secukupnya kepada pembaca agar memahami topik yang akan dibahas.  Saat menuliskan bagian ini, posisikan anda sebagai pembaca – apakah anda tertarik untuk terus membaca?

\section{Rumusan Masalah}

Rumusan Masalah berisi masalah utama yang dibahas dalam tugas akhir. Rumusan masalah yang baik memiliki struktur sebagai berikut:

\begin{enumerate}
    \item Penjelasan ringkas tentang kondisi/situasi yang ada sekarang terkait dengan topik utama yang dibahas tugas akhir.
    \item Pokok persoalan dari kondisi/situasi yang ada, dapat dilihat dari kelemahan atau kekurangannya. Bagian ini merupakan inti dari rumusan masalah.
    \item Elaborasi lebih lanjut yang menekankan pentingnya untuk menyelesaikan pokok persoalan tersebut.
    \item Usulan singkat terkait dengan solusi yang ditawarkan untuk menyelesaikan persoalan.
\end{enumerate}

Penting untuk diperhatikan bahwa persoalan yang dideskripsikan pada subbab ini akan dipertanggungjawabkan di bab Evaluasi apakah terselesaikan atau tidak.

\section{Tujuan}

Tuliskan tujuan utama dan/atau tujuan detil yang akan dicapai dalam pelaksanaan tugas akhir. Fokuskan pada hasil akhir yang ingin diperoleh setelah tugas akhir diselesaikan, terkait dengan penyelesaian persoalan pada rumusan masalah. Penting untuk diperhatikan bahwa tujuan yang dideskripsikan pada subbab ini akan dipertanggungjawabkan di akhir pelaksanaan tugas akhir apakah tercapai atau tidak.

\section{Batasan Masalah}

Tuliskan batasan-batasan yang diambil dalam pelaksanaan tugas akhir. Batasan ini dapat dihindari (tidak perlu ada) jika topik/judul tugas akhir dibuat cukup spesifik.

\section{Metodologi}

Tuliskan semua tahapan yang akan dilalui selama pelaksanaan tugas akhir. Tahapan ini spesifik untuk menyelesaikan persoalan tugas akhir. Tahapan studi literatur tidak perlu dituliskan karena ini adalah pekerjaan yang harus Anda lakukan selama proses pelaksanaan tugas akhir. Bila rumusan masalah berbentuk aksional, cantumkan diagram blok dari sistem. Jika tidak, cantumkan diagram alir. Contoh diagram blok dari sistem ditunjukkan pada Gambar \ref{figure:contoh_diagram_blok_sistem}.

\begin{figure}
	\small
	\centering
	\begin{tikzpicture}[auto, node distance=2cm, >=latex']
		% Boks persepsi
		\node (tekssubsistem1) [] {Subsistem 1};
		\node [block, below of=tekssubsistem1, node distance=1cm] (subsubsistem1) {Subsubsistem 1};
		\node [block, below of=subsubsistem1, node distance=1.5cm] (subsubsistem2) {Subsubsistem 2};
		\node[fit=(tekssubsistem1) (subsubsistem2), dashed,draw,inner sep=0.2cm] (subsistem1) {};
		
		% Boks Navigasi
		\node (tekssubsistem2) [right=1.5cm of tekssubsistem1] {Subsistem 2};
		\node [block, below of=tekssubsistem2, node distance=1cm] (subsubsistem3) {Subsubsistem 3};
		\node [block, below of=subsubsistem3, node distance=1.5cm] (subsubsistem4) {Subsubsistem 4};
		\node[fit=(tekssubsistem2) (subsubsistem4), dashed,draw,inner sep=0.2cm] (subsistem2) {};
		
		% Subsistem lain
		\node [block, left of=subsistem1, node distance=5cm] (subsistem3) {Subsistem 3};
		\node [block, above of=tekssubsistem2] (subsistem4) {Subsistem 4};
		
		\draw[->] (subsistem3) -- node {} (subsistem1);
		\draw[->] (subsistem1) -- node {} (subsistem2);
		\draw[<->] (subsistem1) |- node {} (subsistem4);
		\draw[<->] (subsistem4) -- node {} (subsistem2);
	\end{tikzpicture}
	\caption{Contoh Diagram Blok Sistem}
	\label{figure:contoh_diagram_blok_sistem}
\end{figure}

\section{Sistematika Penulisan}

Subbab ini berisi penjelasan ringkas isi per bab. Penjelasan ditulis satu paragraf per bab buku.

\section{Jadwal Pelaksanaan dan Anggaran}

Subbab ini berisi jadwal pelaksanaan tugas akhir dan anggaran pelaksanaan tugas akhir. Contoh jadwal pelaksanaan tugas akhir ditunjukkan pada Gambar \ref{figure:contoh_jadwal_pelaksanaan} dan contoh anggaran pelaksanaan tugas akhir dirangkum dalam Tabel \ref{table:contoh_anggaran}.

\begin{figure}[h]
	\small
	\centering
	\begin{ganttchart}[
		hgrid,
		vgrid,
		y unit chart=0.5cm,
		y unit title=0.6cm,
		title height=1,
		x unit=1mm,
		time slot format=isodate,
		time slot unit=day]{2020-09-01}{2020-12-31}
		\gantttitlecalendar{year, month, week=1} \\
		\ganttgroup{Grup Aktivitas 1}{2020-09-01}{2020-09-30} \\
		\ganttbar{Aktivitas 1}{2020-09-01}{2020-09-07} \\
		\ganttbar{Aktivitas 2}{2020-09-08}{2020-09-30} \\
		\ganttgroup{Grup Aktivitas 2}{2020-09-21}{2020-12-31} \\
		\ganttbar{Aktivitas 3}{2020-09-21}{2020-10-07} \\
		\ganttbar{Aktivitas 4}{2020-10-15}{2020-11-07} \\
		\ganttbar{Aktivitas 5}{2020-11-15}{2020-12-31} \\
		\ganttgroup{Grup Aktivitas 3}{2020-10-01}{2020-12-31}
	\end{ganttchart}
	\caption{Contoh Jadwal Pelaksanaan Tugas Akhir}
	\label{figure:contoh_jadwal_pelaksanaan}
\end{figure}

\begin{table}[htbp]
	\small
	\centering
	\caption{Anggaran Biaya Pelaksanaan Tugas Akhir}
	\label{table:contoh_anggaran}
	\begin{tabular}{lcrr}
		\toprule
		\multicolumn{1}{l}{\textbf{Hal}} & \multicolumn{1}{l}{\textbf{Satuan}} & \multicolumn{1}{l}{\textbf{Harga Satuan}} & \multicolumn{1}{r}{\textbf{Jumlah}}\\
		\midrule
		\textbf{Grup Keperluan 1} \\
		Keperluan 1 & 1 buah & Rp1.000.000,00 & Rp1.000.000,00 \\
		Keperluan 2 & 1 set & Rp400.000,00 & Rp400.000,00 \\
		\midrule
		\textbf{Grup Keperluan 2} \\
		Keperluan 3 & 1 buah & Rp2.000.000,00 & Rp2.000.000,00 \\
		Keperluan 4 & 2 buah & Rp300.000,00 & Rp600.000,00 \\
		\midrule
		\textbf{Total} & & & Rp4.000.000,00 \\
		\bottomrule
	\end{tabular}
\end{table}

    \chapter{TINJAUAN PUSTAKA}
\label{chapter:2}

Bab Studi Literatur digunakan untuk mendeskripsikan kajian literatur yang terkait dengan persoalan tugas akhir. Tujuan studi literatur adalah:

\begin{enumerate}
    \item menunjukkan kepada pembaca adanya gap seperti pada rumusan masalah yang memang belum terselesaikan,
    \item memberikan pemahaman yang secukupnya kepada pembaca tentang teori atau pekerjaan terkait yang terkait langsung dengan penyelesaian persoalan, serta
    \item menyampaikan informasi apa saja yang sudah ditulis/dilaporkan oleh pihak lain (peneliti/Tugas Akhir/Tesis) tentang hasil penelitian/pekerjaan mereka yang sama atau mirip kaitannya dengan persoalan tugas akhir.
\end{enumerate}

\section{Dasar Teori}
Perujukan literatur dapat dilakukan dengan menambahkan entri baru di berkas. Tulisan ini merujuk pada \parencite{knuth2001art}. Ini adalah contoh penggunaan singkatan yang telah didefinisikan: penggunaan pertama: \gls{utc}, kedua dan selanjutnya: \gls{utc}. Juga dapat digunakan singkatan lain: \gls{adt}, \gls{est}. Berikut adalah contoh penggunaan lambang: \gls{symb:pi}, \gls{symb:avogadro}.

    \subsection{Bekerja dengan Float}

    Float adalah \textit{container} untuk elemen-elemen dokumen yang tidak dapat dipisah menjadi beberapa halaman. Environment ``table'' dan ``figure'' secara default adalah float. Float berguna untuk memudahkan peletakan objek yang tidak cukup jika diletakkan di halaman sekarang. Peletakan float diatur oleh \LaTeX\ dan pengguna sebaiknya memberikan keleluasaan kepada \LaTeX\ agar dapat mengatur peletakan dengan baik. 
    
    \subsubsection{Gambar}
    
    Float bisa di-\textit{cross reference}. Contohnya Gambar~\ref{fig:contoh_gambar} adalah contoh gambar.

    \begin{figure}[h]
        \centering
        \includegraphics[width=0.8\textwidth]{resources/chapter-2-infrastructure-diagram.png}
        \caption[Contoh caption yang muncul di daftar]{Contoh gambar}
        \label{fig:contoh_gambar}
    \end{figure}

    \subsubsection{Tabel}

    Tabel juga merupakan float. Tabel~\ref{table:contoh_tabel} adalah contoh tabel.

    \begin{table}[htbp]
        \small
        \centering
        \caption{Contoh tabel}
        \label{table:contoh_tabel}
        \begin{tabular}{ll}
            \toprule
            \multicolumn{1}{l}{\textbf{Contoh Judul Kolom}} & \multicolumn{1}{l}{\textbf{Nilai}}\\
            \midrule
            Besaran 1 & 12 meter          \\
            Besaran 2 & $360^\circ$       \\
            Besaran 3 & 0,2 meter         \\
            Besaran 4 & $1^\circ$         \\
            Besaran 5 & 8000 sampel/detik \\
            \bottomrule
        \end{tabular}
    \end{table}

	\subsection{Algoritma}
	
	\blindtext Algoritma \ref{algoritma:ekf}  \parencite{leonard1991mobile, thrun2005probabilistic} menunjukkan contoh penulisan algoritma.
	
	\begin{algorithm}
		\caption{Extended Kalman Filter Localization}
		\label{algoritma:ekf}
		\begin{algorithmic}[1]
			\Function{EKFLocalization}{$\mu_{t-1}, \Sigma_{t-1}, u_t, z_t, m$}
			\State $\bar{\mu}_t = $ \textbf{motion\_model}($u_t, \mu_{t-1}$)
			\State $\bar{\Sigma}_t = G_t \Sigma_{t-1} G_t^T + V_t M_t V_t^T$
			\For {all observed features $z_t^i = (r_t^i \ \phi_t^i \ s_t^i)^T$}
			\For {all landmarks $k$ in the map $m$}
			\State $\hat{z}_t^k = $ \textbf{measurement\_model}($\bar{\mu}_t, m$)
			\State $S_t^k = H_t^k \bar{\Sigma}_t [H_t^k]^T + Q_t$ 
			\EndFor
			\State $j(i) = \argmax\limits_{x} \det(2 \pi S_t^k)^{\frac{1}{2}} \exp\{-\frac{1}{2} (z_t^i - \hat{z}_t^k)^T [S_t^k]^{-1} (z_t^i - \hat{z}_t^k)\}$
			\State $K_t^i = \bar{\Sigma}_t [H_t^{j(i)}]^T [S_t^{j(i)}]^{-1}$
			\State $\bar{\mu}_t = \bar{\mu}_t + K_t^i (z_t^i - \hat{z}_t^{j(i)})$
			\State $\bar{\Sigma}_t = (I - K_t^i H_t^{j(i)}) \bar{\Sigma}_t$
			\EndFor
			\State $\mu_t = \bar{\mu}_t$
			\State $\Sigma_t = \bar{\Sigma}_t$
			\State \Return $\mu_t, \Sigma_t$
			\EndFunction
		\end{algorithmic}
	\end{algorithm}	

    \subsection{Persamaan Matematika}

    \blindtext Persamaan~\eqref{eq:contoh_equation} adalah contoh persamaan matematika,
    \begin{align}
        c^2 = a^2 + b^2\,.
    \label{eq:contoh_equation}
    \end{align}
    
    Contoh penggunaan notasi custom,
    \begin{align}
        \bayes{x}{y}\,.
    \label{eq:contoh_equation_custom}
    \end{align}

\section{Studi Terkait}
\blindtext

    \chapter{ANALISIS DAN PERANCANGAN}
\label{chapter:3}

\section{Analisis Masalah}
\blindtext

\section{Solusi Umum}
\blindtext

\section{Rancangan Solusi}
\blindtext
    \chapter{EVALUASI DAN PEMBAHASAN}
\label{chapter:4}

\section{Tujuan Pengujian}
\blindtext

\section{Skenario Pengujian}
\blindtext

\section{Hasil Pengujian}
\blindtext

\section{Pembahasan}
\blindtext
    \chapter{PENUTUP}
\label{chapter:5}

\section{Kesimpulan}
\blindtext

\section{Saran}
\blindtext
    %----------------------------------------------------------------%

    % Daftar pustaka
    %\begingroup
    %    \renewcommand{\baselinestretch}{1.0}
    %    \printbibliography[heading=bibintoc]
    %\endgroup
    
    \clearpage
    \bibliographystyle{IEEEtran}
    \bibliography{references}
    
    % Before:
    % ---
    % Index
    % \appendix
    % \addcontentsline{toc}{part}{Lampiran}
    % \part*{Lampiran}
    % ---

    % Format judul bab lampiran
    \titleformat{\chapter}[hang]
      {\large\bfseries}
      {\chaptertitlename\ \thechapter}{1em}
        {\large\bfseries}
    \titlespacing*{\chapter}{0pt}{-1.5\baselineskip}{\parskip}
    
    \clearpage
    \begin{center}
        \topskip0pt
        \vspace*{\fill}
        \large\textbf{LAMPIRAN}\normalsize
        \vspace*{\fill}
    \end{center}
    \clearpage
    
    \begin{appendices}
        \input{chapters/appendix-1}
        \input{chapters/appendix-2}
    \end{appendices}

\end{document}
